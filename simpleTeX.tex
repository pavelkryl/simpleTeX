% z basemacros prevzit: styly, prostredi, promenne, zahlavi + zapati, barvy, krizove odkazy,
% obrazky, obsah
% ^- dat do baliku

\let\ex\expandafter
\def\YES{YES}%
\def\NO{NO}%
\let\DPRINT\message
% disabling debug messages
\def\DPRINT#1{}%
\def\FatalErrMsg#1{\errmessage{#1}}%

\ex\newtoks\csname @hlp::TmpToks@\endcsname
\def\TmpToksCS{\csname @hlp::TmpToks@\endcsname}%

\def\NewLine{\hfill\break}%
\def\NewPage{\vfill\eject}%
\def\\{\NewLine}

%% variable manipulation

{\catcode`\@=12%
\gdef\VarPrefix{@var@:}%
%% sets the value of variable #1 to value #2
% 1 - variable name
% 2 - variable value
\long\gdef\SetVar#1#2{\ex\gdef\csname\VarPrefix#1\endcsname{#2}}%
%% sets variable locally - after closing the group variable converts to it's
%% original value, params:
% 1 - variable name
% 2 - variable value
\gdef\lSetVar#1#2{\ex\def\csname\VarPrefix#1\endcsname{#2}}%
%% sets variable #1 to expanded value of #2, params:
% 1 - variable name
% 2 - variable value: text to be expanded before assignment
\gdef\eSetVar#1#2{\ex\xdef\csname\VarPrefix#1\endcsname{#2}}%
%% content of source variable is directly copied (no expansion) to destination,
%% this is something else then: \SetVar{Destination}{\GetVar{Source}} because if
%% the source is changed later, then Destination is also affected - this is not
%% this case, params:
% 1 - source variable name
% 2 - destination variable name
\gdef\CopyVar#1->#2{%
	\ex\ex\ex\TmpToksCS\ex\ex\ex=\ex\ex\ex{\csname\VarPrefix#1\endcsname}%
	\eSetVar{#2}{\the\TmpToksCS}%
}%

%% tests whether the given variable has been set to, params:
% 1 - name of the variable
% 2 - true text (then)
% 3 - false text (else)
\gdef\IfSetVar#1#2#3{%
\ex\ifx\csname\VarPrefix#1\endcsname\relax
#3\else
#2\fi}%

%% returns the value of previously set variable
% 1 - variable ?of which value? we are interested in
\gdef\GetVar#1{\csname\VarPrefix#1\endcsname}}%

%% tests whether given variable exists, if it does not, error message is printed
% out an input is awaited, BE CAREFUL, when the variable is set to \relax, it
% behaves as undefined variable, params:
% 1 - variable name
\gdef\TestVarExists#1{\ex\ifx\csname \VarPrefix#1\endcsname\relax\FatalErrMsg{variable #1 does not exist!}\fi}

%% appends content to an already existing content
%% the existing content is verbatim copied - no expansion is performed, params:
% 1 - name of the variable to which append
% 2 - what to append
\gdef\PostfixVar#1#2{%
	\ex\ex\ex\toks0\ex\ex\ex{\csname\VarPrefix#1\endcsname}%
	\toks1={#2}%
	%\message{contents of toks0 is \the\toks0}%
	%\message{contents of toks1 is \the\toks1}%
	\eSetVar{#1}{\the\toks0 \the\toks1}%
}%

\gdef\PrefixVar#1#2{%
	\ex\ex\ex\toks0\ex\ex\ex=\ex\ex\ex{\csname\VarPrefix#1\endcsname}%
	\toks1={#2}%
	\eSetVar{#1}{\the\toks1 \the\toks0}%
}%

%% tries to return values of variable, if the variable has not been defined
%% yet, prints out an error message and terminates (fatal error), params:
% 1 - variable name
% 2 - error message in case 
%\gdef\safeGetVar#1#2{\IfSetVar{#1}{\GetVar{#1}}{\FatalErrMsg{#2}}}%

%% undefs given variable, params:
% 1 - variable to be undefed
\def\UndefVar#1{\ex\global\ex\let\csname\VarPrefix#1\endcsname=\undef}%

%% returns content of variable in given 0-9 toks register
\gdef\GetVarInToks#1#2{%
	\ex\ex\ex\toks#1\ex\ex\ex=\ex\ex\ex{\csname\VarPrefix#2\endcsname}%
}

%% tests contents of variable
% 1 - variable name
% 2 - expected value
% 3 - true text (then)
% 4 - false text (else)
\long\def\IfVar#1='#2'#3#4{{%
	% local part
	\edef\hlpA{\GetVar{#1}}%
	\edef\hlpB{#2}%
	\ifx\hlpA\hlpB\gdef\next{#3}\else\gdef\next{#4}\fi}%
	% global part
	\next
}%

%% compares numeric content of variables, params:
% 1 - variable1
% 2 - operator (<,>,=)
% 3 - value
% 4 - true text
% 5 - false text
\long\def\IfNumVar#1#2#3#4#5{\ex\ifnum\GetVar{#1}#2#3#4\else#5\fi}
%\long\def\IfNumVar#1#2#3#4#5{\ex\ifnum\GetVar{#1}#2#3#4\else#5\fi}

%% tests two strings
% 1 - string 1
% 2 - string 2
% 3 - true text (then)
% 4 - false text (else)
\long\def\IfLit'#1'='#2'#3#4{{%
	% local part
	\edef\hlpA{#1}%
	\edef\hlpB{#2}%
	\ifx\hlpA\hlpB\gdef\next{#3}\else\gdef\next{#4}\fi}%
	% global part
	\next
}%


%% tests whether string #1 expands to YES string, params:
% 1 - string to be expanded
% 2 - then text
% 3 - else text
\long\def\IfYES'#1'#2#3{\IfLit'#1'='YES'{#2}{#3}}%

%% conditionally includes text #2, depending whether any text with identifier
%% #1 has been already included, params:
% 1 - variable name containing #2 if the text in #3 should be included
% 2 - variable value to be contained for inclusion
% 3 - text to be included
% note: you should rather use \IfVar macro
\long\def\CondInclude#1#2#3{\IfVar#1='#2'{#3}%
	% else
	{\DPRINT{ommitting inclusion according to var #1=\GetVar{#1}}}}%

%% includes specified text #2 only once depending on the variable #1, params:
% 1 - variable on which the 'everlasting lock operation' is performed
% 2 - text guarded by the variable
\long\def\OneShotInclude#1#2{\IfVar#1='locked'{\DPRINT{ommiting file #1}}{\SetVar{#1}{locked}#2}}%

%% includes given file only if it wasn't included before
% 1 - name of the file to be included
\long\def\SafeFileInclude#1{\OneShotInclude{#1}{\input #1\space}}%
\def\LoadPackage#1{\SafeFileInclude{#1.pkg}}%

%% includes given file into the current place in document
\def\LoadFile#1{\input #1\space}%

%% new input file representing softly included file
\newread\SoftInputFile
%% tries to open file for writing, if the file does not exists,
% prints out a warning, if it succeeds, the file is just included, params:
% 1 - file to be included
\def\SoftInput#1{\let\next=\relax
	\openin\SoftInputFile=#1 %
	\ifeof\SoftInputFile\message{Warning: cannot open input file #1}%
	\else\closein\SoftInputFile\def\next{\input #1 }\fi
	\next
}

%% common useful macros manipulating mainly with macros again

%% ignores first group (character) 
\def\Ignore#1{}%
%% removes from the given parameter the first inner group (character)
\def\RemoveLeadingGroup#1{\Ignore#1}%

%% saves text of macro #1 (one level expansion) into macro #2, params:
% 1 - original macro name
% 2 - new macro name which will do the same (direct macro name neccessary!!)
\def\SaveMacro#1#2{\let#2=#1}%

%% overwrites current body of parameterless macro #1 by body #2
% 1 - name of the overwritten parameterless macro
% 2 - new body
\def\OverwriteNoParMacro#1#2{%
	% saving original macro
	\ex\SaveMacro\ex#1\ex{\csname @rIg\Ignore#1\endcsname}%
	% redefining macro
	\def#1{#2}%
}%

%% restores macro previously overwritten by OverwriteNoParMacro, params:
% 1 - name of the previously overwritten macro
\def\RestoreMacro#1{\ex\let\ex#1\ex=\csname @rIg\Ignore#1\endcsname}%
%% returns body of the previously overwritten macro #1, params:
% 1 - name of the macro
\def\GetOrigMacro#1{\csname @rIg\Ignore#1\endcsname}%

\newcount\hlpCount
\newdimen\hlpDim
%% a square, params:
% 1 - size
% 2 - space inserted under the box
\def\Square[#1][#2]{\leavevmode\raise#2\hbox to#1{\vrule width#1 height#1 depth0pt}}%
%\hlpDim=#1\advance\hlpDim by #2
%   \vrule width #1 height\the\hlpDim depth -#2}

%% LaTeX logo
\def\LaTeX{L\kern-.36em\raise.5ex\hbox{\sevenrm A}\kern-.12em\TeX}

\long\def\disable#1{}%

%% conditionally perform #2 or #3 depending on following character, params:
% 1 - expected following character
% 2 - true text (character follows)
% 3 - false text (no such character follows)
\def\IfNextChar#1#2#3{%
	\let\expectedNextChar=#1%
	\def\trueText:{#2}\def\falseText:{#3}%
	\futurelet\nextChar\NextCharHandlingRoutine
}

%% helper routine
\def\NextCharHandlingRoutine{%
	\ifx\nextChar\SpaceToken\let\next\RepeatIfNextChar
	\else
		\if\nextChar\expectedNextChar\let\next\trueText
		\else\let\next\falseText
		\fi
	\fi\next:}%

%% helper CS
\futurelet\SpaceToken{ }
%% helper routine
\def\RepeatIfNextChar: {\futurelet\nextChar\NextCharHandlingRoutine}%

\def\emptyCS{}
%% cutting routine
\def\GetDecimalPart[#1.#2]{\if a#2a \else \OmmitTrailingDot#2\fi}%
\def\GetIntegerPart[#1.#2]{#1}

\def\OmmitTrailingDot#1.{#1}
